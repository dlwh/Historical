\documentclass[11pt,a4paper]{article}
\usepackage{acl2010}
\usepackage{times}
\usepackage{amsfonts}
\usepackage{amssymb}
 \usepackage{graphicx}
\usepackage{amsmath}
\usepackage{multirow}
\usepackage{comment}
\usepackage{enumerate}
\usepackage{latexsym}
%\setlength\titlebox{6.5cm}    % Expanding the titlebox

\title{Finding Cognate Groups using Phylogenies}

\author{}

\date{}

\begin{document}
\maketitle
\begin{abstract}
  A central problem in historical linguistics is the identification
  of historically related \emph{cognate} words.  We present a phylogenetic model for
  automatically inducing cognate group structure from unaligned word lists.
  Our model uses
  weighted transducers to represent distributions over unobserved
  forms as well as the transformations from ancestor word to daughter
  word. We also present a novel method for simplifying the complex
  automata created during inference to counteract the otherwise
  exponential growth of message sizes. We test our model on two
  datasets where we significantly outperform a baseline approach.
  Finally, we demonstrate that our automatically induced
  groups can be used to successfully reconstruct ancestral words.
\end{abstract}
\section{Introduction}

A crowning achievement of historical linguistics is the comparative
method \cite{ohala93phonetics}, wherein linguists use word similarity
to elucidate the hidden phonological and morphological processes
which govern historical descent. The comparative method requires
reasoning about three important hidden variables: the overall
\emph{phylogenetic guide tree} among languages, the \emph{evolutionary
parameters} of the ambient changes at each branch, and the \emph{cognate
group structure} that specifies which words share common ancestors.

All three of these variables interact and inform each other, and
so historical linguists often consider them jointly.  However,
linguists are currently required to make qualitative judgments
regarding the relative likelihood of certain sound changes, cognate
groups, and so on.  Recent statistical methods have been introduced
to provide increased quantitative backing to the comparative
method~\cite{oakes00computer,bouchard07probabilistic,bouchard09improved}.
These automated methods, while providing robustness and scale in
the induction of ancestral word forms and evolutionary parameters,
assume that both guide trees and cognate words are already given.
In this work, we address the latter, more severe limitation,
presenting a model in which cognate groups can automatically be
discovered.

Finding cognate groups is not an easy task, because underlying
morphological and phonological changes can obscure relationships
between words, especially for distant cognates, where simple string
overlap is an inadequate measure of similarity.  Some authors have
attempted to automatically detect cognate
words~\cite{lowe94reconstruction,oakes00computer,Kondrak01identifyingcognates,mulloni07automatic},
but these methods typically work on language pairs rather than on
language families.  [more?]  To handle realistic scenarios, it is
necessary to consider multiple languages, and to do so in a model
which couples cognate detection with similarity learning.

In this paper we present a new generative model for the automatic
induction of cognate groups given only (1) a known family tree of
languages and (2) word lists from those languages.  A prior on word
survival generates a number of cognate groups and decides which
modern languages attest each group.  An evolutionary model captures
how each word is generated from its parent word.  Finally, a
permutation model describes how cognate groups are collapsed into
flat word lists.  Inference requires a combination of message-passing
in the evolutionary model and iterative bipartite graph matching
in the permutation model.

% This needs to be here to convince latex to put it on page 2. 
\begin{figure*}
  \centering
  \includegraphics[scale=0.35]{gmodel}
  \caption{(a) The process by which cognate words are generated.
  Here, we show the derivation of Romance language words from their
  respective Latin ancestor. (b) The class of parameterized edit
  distances used in this paper. Each pair of characters has a weight
  $\sigma$ for deletion, and each character has weights $\eta$ and
  $\delta$ for insertion and deletion respectively. (c) A possible
  alignment produced by an edit distance.}
  \label{fig:gmodel}
\end{figure*}

In the message-passing phase, our model encodes distributions over strings as weighted finite
state automata~\cite{mohri09weighted}.  Weighted automata have been
successfully applied to speech processing \cite{Mohri96weightedautomata} and
more recently to morphology \cite{dreyer2009graphical}.  Here, we present
a new method for automatically compressing our message automata in a way that can take into account
prior information about the expected outcome of inference.

[XXX There should be a sentence or two about the results. Intro already too long.]

\section{Datasets}

In this paper, we focus on two datasets. One is an automatically
transcribed word list of 584 cognate sets from three Romance languages
(Portuguese, Italian and Spanish), as well as their common ancestor
Latin~\cite{bouchard07probabilistic}.  The other is a hand-curated
list of nearly 60,000 cognates sets from 64 modern Austronesian
languages created by \newcite{blust93oceanic}. Both datasets are
transcribed into the International Phonetic Alphabet (Romance
automatically, Austronesian manually). 

One core difference between these datasets is that the Romance
dataset is dense while the Oceanic is sparse. Every cognate set in
the Romance dataset has a word for each language, while in the
Oceanic dataset only 1.6\% of possible slots are filled. Moreover,
it is entirely possible that the Oceanic dataset is incomplete:
there may be currently distinct cognate groups that are derived
from a single ancestral word!

\section{Model}

In this section we describe a new generative model for vocabulary lists
in multiple related languages given the phylogenetic relationship
between the languages (their family tree). The generative
process factors into three subprocesses: survival, evolution, and
permutation. Survival dictates, for each cognate group, which
languages have words in that group. Evolution describes the process
by which words are transformed from their parent word. Finally,
permutation describes the ``scrambling'' of the word lists into a flat
order that hides their lineage. Each process is
described graphically in Figure (\ref{fig:gmodel}a), and we present
each in detail in the following subsections.

\subsection{Survival}

First, we choose a number of ancestral cognate groups from an
exponential distribution.  For each cognate group, our generative
process walks down the tree.  At each branch, the word may either
survive or die.  This process is modeled in a ``death tree'' with
variables $S_\ell$ specifying whether or not the word died before
reaching that language. Death at any node in the tree causes all
of that node's descendants to also be dead.  This process captures
that a word is more likely to be found in two sibling languages
(e.g. Spanish and Portuguese) than in two cousin languages
(Portuguese and Italian); more independent death events occur in the latter case. 

%(XXX where to put this:) Also, because we do not know how many
%cognate groups there may be ahead of time, we place a geometric
%prior on the number of unobserved groups. (XXX)

\subsection{Evolution}

Once we know which languages will have an attested word and which
will not, we generate the actual word forms. The evolution component of
the model generates words according to a branch-specific transformation from a node's 
immediate ancestor.  The figure graphically describes our generative
model for three Romance languages: Italian, Portuguese, and Spanish.
In each cognate group, each word $W_\ell$ is generated from its
parent according to a conditional distribution with parameter $\phi_\ell$,
which is specific to that edge in the tree, but shared between all
cognate groups.

In this paper, each $\phi_\ell$ takes the form of a parameterized
edit distance similar to the standard Levenshtein distance.  Arbitrary
models--such as the richer ones in \newcite{bouchard07probabilistic}--could
instead be used, if at increased inferential cost.   The edit
transducers are represented schematically in Figure (\ref{fig:gmodel}b).
Characters $x$ and $y$ are arbitrary characters, and $\sigma(x,y)$
represents the cost of substituting $x$ with $y$.  $\epsilon$
represents the empty character and is used as short hand for insertion
and deletion, which have parameters $\eta$ and $\delta$, respectively.

As an example, see the illustration in Figure (\ref{fig:gmodel}c).
Here, the Vulgar Latin word \textit{cantare} is generated from its
parent form \textit{canere} by a series of edits: 4 matches, 1
substitution (from `e' to `a') and 1 insertion (`t').
The probability of each individual edit is determined by $\phi$.
Note that the marginal probability of a specific Vulgar Latin word
conditioned on its Latin parent is the sum over all possible
alignments that generate it.

\subsection{Permutation}

Finally, at the leaves of the trees are the observed words. (We
take interior nodes to be unobserved.) Here, we make the (false)
simplifying assumption that in any language there is at most one
word per language per cognate group. Because the assignments of
words to cognates is unknown, we specify an unknown parameter
$\pi_\ell$ for each modern language that specifies a permutation
of the words which specifies the mapping. From a generative
perspective, $\pi_\ell$ generates observed positions of the words
in some vocabulary list.

In this paper, our task is primarily to learn the permutations
$\pi_\ell$. All other hidden variables are auxiliary and are to be marginalized to the greatest extent possible.

\section{Inference}
In this section, we discuss the inference method for determining
cognate assignments under fixed parameters $\phi$.  We are given a
set of languages and a list of words in each language, and our
objective is to determine which words are cognate with each other
words. In the Romance dataset, we have the additional constraint
that each cognate group supports exactly one word from each language
(full cognate groups), while in the Austronesian dataset we have
at most one word from each language (partial groups). In effect,
the inference task is reduced to finding a permutation $\pi$ of the
respective word lists to maximize the log probability of the observed
words:
\begin{equation}
  \begin{split}
    \vec{\pi} = \arg\!\max_{\vec \pi} \sum_{g} \log p(\vec w_{(\ell,\pi_\ell(g))}|\vec \phi,\vec \pi)
   \end{split}
 \end{equation}
Maximizing this equation directly is intractable, and so instead
we use a coordinate ascent algorithm to iteratively maximize one
permutation while holding the others fixed:
\begin{equation}
  \begin{split}
    \pi_\ell = \arg\!\max_{\pi_\ell} \sum_{g} \log p(\vec w_{(\ell,\pi_\ell(g))}|\vec \phi,\vec \pi_{-\ell},\pi_\ell)
  \end{split}
\end{equation}
Each iteration is then actually an instance of bipartite graph
matching, with the words in one language one set of nodes, and the
current cognate groups in the other languages the other set of
nodes, and the edge affinities $\mathrm{aff}$ between these nodes are the conditional
probability of each word $w_\ell$ belonging to each cognate group $g$:
\begin{equation}
  \begin{split}
    \mathrm{aff}(w_\ell,g) = p(w_\ell|\vec w_{-\ell,\pi_{-\ell}(g)},\vec \phi,\vec\pi_{-\ell})
   \end{split}
 \end{equation}

To compute these affinities for each cognate group, inference in each
tree computes the marginal distribution of the words from the ``held
out'' language. For the marginals, we use an analog of the
forward/backward algorithm. In the upward pass, we send messages
from the leaves of the tree toward the root. For observed leaf nodes
$W_d$, we have:
\begin{equation}
  \begin{split}
    \mu_{d\to a}(w_a) = \sum_{w_d} p(w_d|w_a,\phi)
   \end{split}
 \end{equation}
and for interior nodes $W_i$:
\begin{equation}
  \label{eqn:summing}
  \begin{split}
    \mu_{i\to a}(w_a) = \sum_{w_i} p(w_i|w_a) \prod_{d \in \mathrm{child}(w_i)} \mu_{d \to i}(w_i) 
  \end{split}
\end{equation}
In the downward pass (toward the held-out language), we sum over ancestors word $W_a$:
\begin{equation}
  \begin{split}
    \mu_{a\to d}(w_d) = \sum_{w_a} p(w_d|w_a) \prod_{\stackrel{d' \in \mathrm{child}(w_a)}{d' \neq d}} \mu_{d' \to a}(w_a) 
  \end{split}
\end{equation}
Computing these messages gives a posterior marginal distribution
$p(w_\ell|\vec w_{-\ell,\pi_{-\ell}(g)},\vec \phi,\vec\pi_{-\ell})$
over the held-out language, which is precisely the affinity score
we need for the bipartite matching. We then use the Kuhn-Munkres
algorithm \cite{Kuhn1955} to find the optimal assignment for the
bipartite matching problem.

One important final note is initialization. In our early experiments
we found that choosing a random starting configuration unsurprisingly led
to rather poor local optima. Instead, we started with empty trees,
and added in one language per iteration until all languages were
added, and then continued iterations on the full tree.

\section{Learning}

So far we have only addressed searching for Viterbi permutations
$\pi$ under fixed parameters. However, we also have two other sets
of parameters: the transducers $\phi$, and the death probabilities
$S$. The parameters can be learned through standard maximum likelihood
estimation, which we detail in this section.

Estimating the values for each survival parameter $S_\ell$ is
straightforward. Because we enforce that a word must be dead if its
parent word is dead, we just need to learn the conditional probabilities
$p($death$|$parent not dead$)$. Given fixed assignments $\pi$, we
simply count the number of ``deaths'' that occurred under a live
parent, apply smoothing--we found adding 0.5 to be reasonable--and
divide by the total number of live parents.

For the edit transducers $\phi$, we learn parameterized edit distances.  For each $\phi_\ell$ we fit a non-uniform
substitution, insertion, and deletion matrix $\sigma(x,y)$. These edit distances
define a conditional exponential family distribution when conditioned
on an ancestral word (or when multiplied by a prior distribution
over those words). That is, for any fixed $w_a$:
\begin{equation}
  \begin{split}
    \sum_{w_d} &p(w_d|w_a,\sigma) = \sum_{w_d} \sum_{\stackrel{z\in}{\scriptscriptstyle\mathrm{align}(w_a,w_d)}} \mathrm{score}(z;\sigma) \\
    &= \sum_{w_d} \sum_{\stackrel{z\in}{\scriptscriptstyle\mathrm{align}(w_a,w_d)}} \exp( \sum_{(x,y)\in z} \sigma(x,y)) \\
     &= 1
   \end{split}
 \end{equation}
where $\mathrm{align}(w_a,w_d)$ is the set of possible alignments between words $w_a$ and $w_d$.

To construct the transducers, we first compute the marginal
distribution over the edges connecting any two languages $a$ and
$d$. With this distribution, we calculate the expected ``alignment
unigrams.'' That is, we need to find for each pair of characters
$x$ and $y$ the quantity:
\begin{equation}
  \begin{split}
    E_{p(w_a,w_d)}&[\#(x,y)] \\ = \sum_{w_a,w_d} &\sum_{\stackrel{z\in}{\scriptscriptstyle\mathrm{align}(w_a,w_d)}} \#(x,y) p(z|w_a,w_d)p(w_a,w_d)
   \end{split}
 \end{equation}
where we denote $\#(x,y)$ to be the number of times the pair of characters
appears in the pair of strings. The exact method for computing
these counts is to use an expectation semiring described in \newcite{eisner2001expectation}.

Given the expected counts, we now need to normalize them to ensure
that the transducer represents a conditional probability distribution.
Based on the derivation by \newcite{Oncina20061575}, we have that,
for each character $x$ in the ancestor language:
\begin{equation}
  \begin{split}
    \sum_{y \in \Sigma \cup \{\epsilon\}} \sigma(x,y) &= 1 - \sum_{y \in \Sigma} \sigma(\epsilon,y) \\
    \sigma(x,y) &\propto \frac{E[\#(x,y)]}{E[\#(x,\cdot)]} \\
    \sigma(\epsilon,y) &= \frac{E[\#(\epsilon,y)]}{E[\#(\cdot,\cdot)]} \\
   \end{split}
 \end{equation}
These equations ensure that the three transition types (substitution/match,
deletion, and insertion) are normalized for any ancestral character.

\section{Transducers and Approximations}

In our model, it is not just the edit distances that are finite
state machines. Indeed, the words themselves are complex string-valued
random variables. To represent distributions and messages over these
variables, we chose weighted finite state automata, which can
compactly represent functions over strings. Unfortunately, while
initially compact, these automata become unwieldy in inference,
and so approximations must be used.

In this section, we summarize the algorithms and representations
used for weighted finite state transducers.\footnote{For more
detailed treatment of the general transducer operations, we direct
readers to \newcite{mohri09weighted}.} We also describe a new
procedure for approximating automata in a message-passing environment.

\subsection{Weighted Automata}

Informally, a weighted automaton encodes a distribution over strings
(or pairs of strings) as weighted paths through a directed graph.
Each edge in the graph has a label (a single character, or the empty
character) and a weight. The weight of a string is then the sum
of all paths through the graph that accept a string.

More specifically, a weighted transducer assigns real-valued
weights\footnote{Strictly, the weights can be anything that form a
semiring, but we for the sake of exposition we leave we specialize
to reals.} to pairs of strings with characters in alphabets $\Sigma$
and $\Delta$. Each path through the transducer corresponds to an
alignment between two strings, with an associated score for that
alignment. (Weighted automata can be thought of as transducers which
assign score 0 to pairs of strings which are not identical.)

For our purposes, we are concerned with three fundamental operations
on weighted transducers. The first is computing the sum of all paths
through a transducer, which corresponds to computing the partition
function of a distribution over strings. This operation can be
performed in worst-case cubic time (using a generalization of the
Floyd-Warshall algorithm).  For acyclic or feed-forward transducers
this can be improved dramatically by using a generalization of
Djisktra's algorithm, the Bellman-Ford algorithm or other related
algorithms~\cite{mohri09weighted}.

The second operation is the composition of two transducers.
Intuitively, composition creates a new transducer that takes the
output from the first transducer, processes it through the second
transducer, and then returns the output of the second transducer.
That is, consider two transducers $T_1$ and $T_2$. $T_1$ has input
alphabet $\Sigma$ and output alphabet $\Delta$, while $T_2$ has
input alphabet $\Delta$ and output alphabet $\Omega$. The composition
$T_1 \circ T_2$ returns a new transducer over $\Sigma$ and $\Omega$
such that $(T_1 \circ T_2)(x,y) = \sum_{u} T_1(x,u)\cdot T_2(u,y)$.
In this paper, we use composition for marginalization and factor
product. Given a factor $f_1(x,u;T_1)$ and another factor $f_2(u,y;T_2)$,
composition corresponds to the operation $\psi(x,y) = \sum_u f_1(x,u)
f_2(u,y)$. For two messages $\mu_1(w)$ and $\mu_2(w)$, the same
algorithm can be used to find the product $\mu(w) = \mu_1(w)\mu_2(w)$.

The third operation is transducer minimization. Transducer composition
produces $O(nm)$ states, where $n$ and $m$ are the number of states
in each transducer. Repeated compositions compound the problem:
iterated composition of $k$ transducers produces $O(n^k)$ states.
Minimization alleviates this problem by collapsing indistinguishable
states into a single state. Unfortunately, minimization does not
always collapse enough states. In the next section we discuss approaches
to ``lossy'' minimization that produce automata that are not exactly
the same but are much smaller.

\subsection{Message Approximation}
\label{sec:approx}

Recall that in inference, when summing out interior nodes $w_i$ we
calculated the product over incoming messages $\mu_{d\to i}(w_i)$
(Equation \ref{eqn:summing}), and that these products are calculated
using transducer composition. Unfortunately, the maximal number of
states in a message is exponential in the number of words in the
cognate group. Minimization can only help so much: in order for two
states to be collapsed, the distribution over transitions from those
states must be indistinguishable, or indistinguishable to within
some tolerance. In practice, for the automata generated in our
model, minimization removes at most half the states, which is not
sufficient to counteract the exponential growth. Thus, we need to
find a way to approximate a message $\mu(w)$ using a simpler automata
$\tilde\mu(w;\theta)$ taken from a restricted class parameterized
by $\theta$.

In the context of transducers, previous authors have focused on a
combination of n-best lists and unigram\footnote{Here, and for the
rest of the paper, the ``grams'' of interest are characters in the
International Phonetic Alphabet.} back-off
models~\cite{dreyer2009graphical}.  For their problem, n-best lists
are sensible: their nodes' local potentials already focus messages
on a small number of hypotheses.  In our setting, however, n-best
lists are problematic; early experiments showed that a 10,000-best
list for a typical message only accounts for 50\% of message log
perplexity. That is, the posterior marginals in our model are (at
least initially) fairly flat.

% Put this here to make it come before the section heading Message Topologies
\begin{figure*}
  \centering
  \includegraphics[scale=0.4,height=2.1in]{fsa}
  \caption{Various topologies for approximating topologies: (a) a
  unigram model, (b) a bigram model, (c) the anchored unigram model,
  and (d) the n-best plus backoff model used in
  \newcite{dreyer2009graphical}. In (c) and (d), the relative height
  of arcs is meant to convey approximate probabilities.}
  \label{fig:fsa}
\end{figure*}

An alternative approach might be to simply treat messages as
unnormalized probability distributions, and to minimize the KL
divergence between some approximating message $\tilde\mu(w)$ and
the true message $\mu(w)$.  However, messages are not always
probability distributions and--because the number of potential
strings is in principle infinite--they need not sum to a finite
number.\footnote{In an extreme case, suppose we have observed that
$W_d=w_d$ and that $p(W_d=w_d|w_a)=1$ for all ancestral words $w_a$.
Then, clearly $\sum_{w_d} \mu(w_d) = \sum_{w_d} \sum p(W_d=w_d|w_a)
= \infty$ whenever there are an infinite number of possible ancestral
strings $w_a$.} Instead, we propose to minimize the KL divergence
between the ``expected'' marginal distribution and the approximated
``expected'' marginal distribution:
\begin{equation}
  \begin{split}
    \hat\theta &= \arg\!\min_{\theta} D_{KL}(\tau(w)\tilde\mu(w;\theta)||\tau(w)\mu(w) ) \\
    &= \arg\!\min_{\theta} \sum_w \tau(w) \tilde\mu(w;\theta) \log \frac{\tau(w)\tilde\mu(w;\theta)}{\tau(w)\mu(w)} \\
    &= \arg\!\min_{\theta} \sum_w \tau(w) \tilde\mu(w;\theta) \log \frac{\tilde\mu(w;\theta)}{\mu(w)} \\
   \end{split}
 \end{equation}
where $\tau$ is a prior distribution acting as a surrogate for the
posterior distribution over $w$ without the information from $\mu$.
That is, we seek to approximate $\mu$ not on its own, but as it
functions in an environment representing its final context.

In this paper, $\mu(w)$ is a complex automaton with potentially
many states, $\tilde\mu(w)$ is a simple parametric automaton with
forms that we discuss below, and $\tau(w)$ is an arbitrary (but
hopefully fairly simple) automaton. The actual method we use is as
follows. Given a deterministic prior automaton $\tau$, and a
deterministic automaton topology $\tilde\mu^*$, we created the
composed unweighted automaton $\tau \circ \tilde\mu^*$, and calculate
arc transitions weights to minimize the KL divergence between that
composed transducer and $\tau\circ\mu$. The procedure for calculating
these statistics was described in \newcite{li2009first}, which
amounts to using an expectation semiring \cite{eisner2001expectation}
to compute expected transitions in $\tau\circ\tilde\mu^*$ under the
probability distribution $\tau\circ\mu$.

From there, we need to create the automaton $\tau^{-1}
\circ\tau\circ\tilde\mu$. That is, we need to divide out the influence
of $\tau(w)$. Since we know the topology and arc weights for $\tau$
ahead of time, this is often as simple as dividing arc weights in
$\tau\circ\tilde\mu$ by the corresponding arc weight in $\tau(w)$.
For example, if $\tau$ encodes a geometric distribution over word
lengths and a uniform distribution over characters (that is, $\tau(w)
\propto {p^{|w|}}$), then computing $\tilde\mu$ is as simple as
dividing each arc in $\tau\circ\tilde\mu$ by $p$.\footnote{Actually,
we must be sure to divide each final weight in the transducer
by $(1-|\Sigma| p)$, which is the stopping probability for a geometric
transducer.}

There are a number of choices for $\tau$. One is a hard maximum on
the length of words. Another is to choose $\tau(w)$ to be a unigram
language model over the language in question with a geometric
probability over lengths. In our experiments, we find that $\tau(w)$
can be a geometric distribution over lengths with a uniform
distribution over characters and still obtain reasonable results.
This distribution captures the importance of shorter strings while
still maintaining a relatively weak prior.

\subsection{Message Topologies}

What remains is the selection of the topologies for the approximating
message $\tilde\mu$. We consider three possible approximations,
illustrated in Figure~\ref{fig:fsa}. The first is a plain unigram
model, the second is a bigram model, and the third is an anchored
unigram topology: a position-specific unigram model for each position
up to some maximum length.

The standard unigram model has $|\Sigma|+2$ parameters:
one weight $\sigma_a$ for each character $a \in \Sigma$, a ``starting
weight'' $\lambda$, and a stopping probability. Estimating this
model involves only computing the expected count of each character,
along with the expected length of a word, $E[|w|]$. We then normalize
the counts according to the maximum likelihood estimate, which
gives: 
\begin{equation*}
  \begin{split}
    \sum_{a\in\Sigma} \sigma_a &= \frac{E[|w|]-1}{E[|w|} \\
    \sigma_a &\propto E[\#(a)]
   \end{split}
 \end{equation*}
with the stop probability taking the remaining mass. Recall that
these expectations can be computed using an expectation semiring.

$\lambda$ can be computed by ensuring that the
approximate and exact expected marginals have the same partition
function. That is, with the other parameters fixed solve:
\begin{equation*}
  \begin{split}
    \sum_w \tau(w) \tilde\mu(w) = \sum_w \tau(w) \mu(w)
  \end{split}
\end{equation*}

The second topology we consider is the bigram topology, which is
similar to the unigram topology except that, instead of a single
state, we have a state for each character in $\Sigma$, along with
a special start state. Each state $a$ has transitions with weights
$\sigma_{b|a}= p(b|a)$. Normalization is straightforward:
\begin{equation*}
  \begin{split}
    \sum_{b\in\Sigma} \sigma_{b|a}&= 1-\sigma_{\mathrm{stop}|a} \\
    \sigma_{b|a} &\propto E[\#(b|a)]
   \end{split}
 \end{equation*}

The final topology we consider takes positional information into
account. Namely, for each position (up to some maximum position),
we have a unigram model over characters emitted at that position,
along with the probability of stopping at that position (i.e. a ``sausage lattice''). Estimating
the parameters of this model is similar, except that the expected
counts for the characters in the alphabet are conditioned on their
position in the string. With the expected counts for each position,
we normalize each state's final and outgoing weights. In our
experiments, we set the maximum length to 7 + the
longest observed string.

\section{Experiments}

We conduct three experiments. The first is a ``complete data''
experiment, in which we reconstitute the cognate groups from the
Romance data set, where all cognate groups have words in all three
languages.  This task highlights the evolution and permutation
models.  The second is a much harder ``partial data'' experiment,
in which we examine a subset of 10 languages in the Oceanic dataset.
Here, only a fraction of words appear in any cognate group, so this
task crucially involves the survival model.  The ultimate purpose
of the induced cognate groups is to feed richer evolutionary models,
such as full reconstruction models.  Therefore, we also consider a
proto-word reconstruction experiment.  For this experiment, using
the system of \newcite{bouchard09improved}, we compare the
reconstructions produced from our automatic groups to those produced
from gold cognate groups.

\begin{table*}
  \centering
  \begin{tabular}{|c|c|c|c||c|c|c|c||}
    \hline
    \multicolumn{2}{|c|}{\phantom{aa}} &\multicolumn{2}{|c||}{Romance} & \multicolumn{4}{|c||}{Austronesian}\\
    \hline
    Transducers & Messages & P. Acc. & Rec. Acc. & P. Prec. & P. Rec. & P. F1 & Rec. Acc\\
    \hline
    N/A & Baseline & 48.1 & 35.4  & & & &\\
    \hline
    Levenshtein&Unigrams & 37.2 & 26.2 & & & &\\
    Levenshtein&Bigrams & 43.0 & 26.5 & & & &\\
    Levenshtein&Anch. Unigrams & 68.6 & 56.8& & & &\\
    \hline 
    Learned&Unigrams & 0.1 & 0.0 & & & &\\
    Learned&Bigrams & 38.7 & 11.3 & & & &\\
    Learned&Anch. Unigrams & \textbf{90.3}  & \textbf{86.6} & & & &\\
    \hline
  \end{tabular}
  \caption{Accuracies for reconstructing cognate groups. Levenshtein
  refers to fixed parameter edit distance transducer. Learned refers
  to automatically learned edit distances. ``P.'' means macro-averaged
  pairwise, ``Rec.'' refers to percentage of completely and accurately
  reconstructed groups.}
  \label{tbl:exp1}
\end{table*}

\subsection{Baseline}

As a baseline for cognate group detection, we use an iterative
bipartite matching algorithm where instead of conditional likelihoods
for affinities we use Dice's coefficient, defined, for sets X and
Y, as:
\begin{equation}
  \begin{split}
    \mathrm{Dice}(X,Y) &= \frac{2 |X\cap Y|}{|X| + |Y|}
   \end{split}
 \end{equation}
Dice's coefficients are commonly used in bilingual detection of
cognates~\cite{Kondrak01identifyingcognates,Kondrak03cognatescan}.We
follow prior work and use sets of bigrams within words. In our case, during
bipartite matching the set X is the set of bigrams in the held-out
language, and Y is the union of bigrams in the other languages.
When the number of cognate groups is not known, we XXX (not sure
still\dots)

\subsection{Experiment 1: Complete Data}

In this experiment, we know precisely how many cognate groups there
are and that every cognate group has a word in each language. This
is, of course, an unrealistic scenario, but it represents a good
test case of how well these models can perform without the
non-parametric task of decided how many clusters to use.

We scrambled the 584 cognate groups in the Romance dataset and ran
the algorithm to convergence. Besides the baseline, we tried using
Unigrams, Anchored Unigrams, and Bigrams with and without learning
the parametric edit distances. When we did not use learning, we set
the parameters of the edit distance to (0, -3, -4) for matches,
substitutions, and deletions/insertions, respectively. With learning
enabled, transducers were initialized with these parameters.

For evaluation, we report two metrics. The first is pairwise accuracy
for each pair of languages averaged across pairs. The other is
accuracy measured in terms of the number of correctly and completely
reconstructed cognate groups.

The left half of Table \ref{tbl:exp1} shows the results under various
configurations. As can easily be seen, the kind of approximation
used matters quite a lot. In this application, positional information
is extraordinarily important, more so than the context of the
previous character. Both Unigrams and Bigrams significantly
under-perform the baseline, while Anchored Bigrams easily outperforms
it both with and without learning.

An initially surprising result is that learning actually harms
performance under the unanchored approximations. The explanation
is that the topologies are not sensitive enough to context, and
that the learning procedure ends up flattening the distributions.
In the case of unigrams--which has the least context--learning
reduces performance to chance.

\subsection{Experiment 2: Austronesian}

XXX

\subsection{Experiment 3: Reconstructions}

XXX

\section{Conclusion}

In this paper, we presented a new generative model of word lists
that automatically finds cognate groups from scrambled vocabulary
lists. This model jointly models the origin, propagation, and
evolution of cognate groups from a common root word. Using the new
approximation technique we introduced in this paper, our model can
reduce the error rate by 80\% over a baseline approach inspired by
previous work. XXX a few more sentences about other results.

Future directions for cognate identification include adding in
semantic information and improving the reconstruction components
for joint identification and reconstruction. \newcite{Kondrak03cognatescan}
showed impressive gains on a bilingual cognate identification task
using semantic distances based on English glosses in the word lists.
Improving the reconstruction model could focus on using improved
edit distance models (e.g. \newcite{bouchard09improved}), or further
iterating on the approximations introduced in this paper.  Another
important direction is to extract cognates from raw text, which
would most likely require easing the bipartite assumption.

XXX (conclude conclusively)

\bibliographystyle{acl}
\bibliography{refs}


\end{document}

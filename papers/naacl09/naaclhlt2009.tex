%
% File naaclhlt2009.tex
%
% Contact: nasmith@cs.cmu.edu

\documentclass[11pt]{article}
\usepackage{naaclhlt2009}
\usepackage{times}
\usepackage{amsfonts}
\usepackage{amssymb}
\usepackage{amsmath}
\usepackage{multirow}
\usepackage{enumerate}
\usepackage{latexsym}
\setlength\titlebox{6.5cm}    % Expanding the titlebox

\title{}

\author{}

\date{}

\begin{document}
\maketitle
\begin{abstract}
\end{abstract}
\section{Introduction}
\section{Model}

We make a few high-level modeling assumptions. First, all words in
a cognate group have a single common ancestor word in a single
ancestral language. Second, each surface form in each language
evolved from its parent form via a regular stochastic process that
is shared across all cognate groups. Specifically, each branch in
the phylogenetic tree has a weighted finite state transducer from
parent strings to child strings.

With these principles in mind, we define a generative model of the cognate groups and their individual word forms in each language as follows:
\begin{enumerate}
  \item Choose a (binary) tree topology over languages.
  \item For each edge in the tree, choose a transducer $\phi_{a,d}$ over strings in an ancestor language $a$ and descendent language $d$.
  \item For each cognate group $g$:
    \begin{enumerate}
      \item Choose a root word form $a_{g\mathrm{ROOT}}$ uniformly over strings. (XXX This is degenerate. Oh well.)
      \item For each interior child $d$ with ancestor $a$:
        \begin{enumerate}
          \item Generate ancestral word $a_{gd} \sim \phi_{a,d}(a_{ga},a_{gd})$.
        \end{enumerate}
    \end{enumerate}
  \item For each word position $i$ in language $\ell$:
    \begin{enumerate}
      \item Choose a cognate group $z_{\ell i}$ uniformly at random. (XXX: will probably want a DP over this)
      \item Choose $w_{\ell i} \sim \phi_{Pa(\ell),\ell}(a_{z_{\ell i}Pa(\ell)},w_{\ell i})$.
    \end{enumerate}
\end{enumerate}

In our setting, we observe only the binary tree topology and the words in each language, but not their ancestral forms, or even which cognate group they belong to. Estimating
the remaining parameters requires EM, which we derive from a variational prospective. The full data likelihood is given by:
\begin{equation}
  \begin{split}
    p(\vec w, \vec z, \vec a, \vec \phi) &= (\prod_\ell \prod_i p(w_{\ell i}|g_{\ell i},w_{g_{\ell i}\ell},\phi_{Pa(\ell),\ell})p(g_{\ell i}))\\
      & (\prod_g p(\vec a_g|\vec\phi) ) p(\vec \phi)
   \end{split}
 \end{equation}
 For simplicity, we assume (XXX probably won't want to do this) a uniform distribution over cognate group assignments $p(g_{\ell i })$, and a degenerate prior over
 transducers.
 We consider the factorized distribution $q(\vec z, \vec a, \vec \phi) = (\prod_{a_1,a_2} q_{a_1,a_2}(\vec \phi_{a_1,a_2}))\prod_\ell \prod_i q_{\ell i}(z_{\ell i})\prod_g
 q_g(\vec a_g)$, where we restrict the $q(\phi)$'s to be point-masses. We seek to minimize the KL divergence between $q$ and the posterior distribution $p(\vec z, \vec a, \vec \phi| \vec w)$.  Doing so leads to the
 updates:
 \begin{equation}
   \begin{split}
     \log q_{\ell i}(z_{\ell i}) &\propto E_q[\log p(w_{\ell i}, z_{\ell i}, a_{z_{\ell i} Pa(\ell)},\vec \phi)] \\
     &\propto E_q[\log  p(w_{\ell i}|g_{\ell i},w_{g_{\ell i}\ell},\phi_{Pa(\ell),\ell}) ]
     &\propto \sum \phi^*(w_{\ell i}) q(a_{gPa(\ell)}) \\
     &\propto \sum 
    \end{split}
  \end{equation}
\section{Experiments}
\section{Conclusion}
\end{document}
